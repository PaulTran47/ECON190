%
% File: chap03.tex
% Author: Derrick Choe, Paul Le Tran
% Description: Chapter about discussing the results of the tuberculosis model and its findings and implications.
%
\let\textcircled=\pgftextcircled
\chapter{Discussion}
\label{chap:4}

\section{Industry Concentration \& Profitability}

\initial{O}ur results point towards a strong relationship between industry concentration and profitability amongst non-manufacturing industries. With 1\% increases in a non-manufacturing industry’s top 50 firm revenue share corresponding to an amount between \$4.7 to \$8.9 million increases in median gross profits, and a host of literature arguing for market concentration's effect on profitability through market power, we argue for the possibility of a causal relationship between our variables. However, we acknowledge the plausibility of a degree of reverse causality within our model. If profitable firms are able to influence market characteristics and negatively impact competition, we cannot interpret Model (2)’s coefficients as an accurate measure of the profit effect of a change in industry concentration. \\

	It should be noted that in manufacturing industries, the relationship between concentration and profitability is obscured and weaker. We fail to determine that a 1\% increase in value added concentration in manufacturing firms has a statistically significant, positive effect on median gross profits. This discrepancy in our results may exist for several reasons. First, it is possible that differences in our methods for measuring market concentration affect our ability to consistently approximate market power in an industry. In our analysis, we find that market revenue concentration is strongly correlated with profitability, while the percentage of value added by the top 50 firms in an industry is not. These results are somewhat surprising-- value added measures the amount of additional money a firm can earn selling a product after deducting labour, service, raw input costs. We expect that industries where a select few firms hold the majority of value added would be highly profitable. Still, our analysis contradicts this notion. \\

We additionally explore the possibility that manufacturing industries, by means of their operations, may be less able to actualize profits through market power. Manufacturing deals exclusively in the process of utilizing raw materials to create a product; input price transparency may diminish a firm’s ability to charge a mark-up. As such, we surmise that manufacturing firms may be less able to exert market power and influence prices to improve profitability. \\

	Naturally, we shift our focus towards economic theory and the implementation of instrumental variables in order to isolate the effect of changing market conditions on firm profitability. We will review existing literature on the subject, looking for economic and empirical arguments for the relationship between industry concentration and profit. In addition, we will investigate potential exogenous variables in order to implement two-stage least squares regression, with the goal of establishing a causal relationship between market competition and profitability. Finally, we will address how our results fit into a broader macroeconomic context, in which increasing market power and profitability may be negatively impacting consumer welfare, labour’s share of income, and economic growth. \\

\section{The Profit Theory of Investment}

\initial{W}hile our results in Section 4.1 are compelling and indicate a clear direction for further investigation, our test of the Profit Theory of Investment fails to confirm a strong relationship between firm profits and investment. Measuring investment as both capital expenditures and research and development expenditures, we find that increases in growth in gross profits only correspond to small changes in investment's growth (with changes ranging from 9-14\% for every 100\% increase in profitability). Our analysis indicates that, taking into account industry characteristics, firm size, and other controls, a company’s investment is not strongly related to its profitability. We conclude that from the time period of 1950-2016, investment levels in the United States are not primarily the result of firm profitability. \\

Although our results do not confirm our initial suspicions that market concentration, profits, and investment are strongly  intertwined through the Profit Theory of Investment, we are still eager to investigate how investment in the United States has evolved over the past few decades. Although investment may not be strongly influenced by a firm’s available internal funds, it is important to seek other mechanisms through which firms might be encouraged to invest. \\

\section{Conclusion}


\initial{I}t is clear that companies are becoming larger, more profitable, and more powerful; through our analysis, we find that this increase in profits does not result in a substantial increase in investment. As such, we cannot ignore the potential issue that increasing competition leads to stagnation in investment and innovation, as suggested by \cite[Alesina et al., 2005]{Alesina2005}. We believe that it is prudent to further investigate the macroeconomic implications of increasing market concentration. For now, the state of investment in the U.S. in unclear: We hope to better understand the future of technological progress, innovation, and productivity growth under current market conditions.

%=========================================================